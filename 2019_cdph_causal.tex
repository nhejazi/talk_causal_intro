\PassOptionsToPackage{unicode=true}{hyperref} % options for packages loaded elsewhere
\PassOptionsToPackage{hyphens}{url}
%
\documentclass[ignorenonframetext,]{beamer}
\usepackage{pgfpages}
\setbeamertemplate{caption}[numbered]
\setbeamertemplate{caption label separator}{: }
\setbeamercolor{caption name}{fg=normal text.fg}
\beamertemplatenavigationsymbolsempty
\usepackage{lmodern}
\usepackage{amssymb,amsmath}
\usepackage{ifxetex,ifluatex}
\usepackage{fixltx2e} % provides \textsubscript
\ifnum 0\ifxetex 1\fi\ifluatex 1\fi=0 % if pdftex
  \usepackage[T1]{fontenc}
  \usepackage[utf8]{inputenc}
  \usepackage{textcomp} % provides euro and other symbols
\else % if luatex or xelatex
  \usepackage{unicode-math}
  \defaultfontfeatures{Ligatures=TeX,Scale=MatchLowercase}
\fi
\usetheme[]{metropolis}
\usefonttheme{structurebold}
% use upquote if available, for straight quotes in verbatim environments
\IfFileExists{upquote.sty}{\usepackage{upquote}}{}
% use microtype if available
\IfFileExists{microtype.sty}{%
\usepackage[]{microtype}
\UseMicrotypeSet[protrusion]{basicmath} % disable protrusion for tt fonts
}{}
\IfFileExists{parskip.sty}{%
\usepackage{parskip}
}{% else
\setlength{\parindent}{0pt}
\setlength{\parskip}{6pt plus 2pt minus 1pt}
}
\usepackage{hyperref}
\hypersetup{
            pdftitle={Causal Inference in Biomedical Research},
            pdfauthor={Nima Hejazi, UC Berkeley},
            pdfborder={0 0 0},
            breaklinks=true}
\urlstyle{same}  % don't use monospace font for urls
\newif\ifbibliography
% Prevent slide breaks in the middle of a paragraph:
\widowpenalties 1 10000
\raggedbottom
\setbeamertemplate{part page}{
\centering
\begin{beamercolorbox}[sep=16pt,center]{part title}
  \usebeamerfont{part title}\insertpart\par
\end{beamercolorbox}
}
\setbeamertemplate{section page}{
\centering
\begin{beamercolorbox}[sep=12pt,center]{part title}
  \usebeamerfont{section title}\insertsection\par
\end{beamercolorbox}
}
\setbeamertemplate{subsection page}{
\centering
\begin{beamercolorbox}[sep=8pt,center]{part title}
  \usebeamerfont{subsection title}\insertsubsection\par
\end{beamercolorbox}
}
\AtBeginPart{
  \frame{\partpage}
}
\AtBeginSection{
  \ifbibliography
  \else
    \frame{\sectionpage}
  \fi
}
\AtBeginSubsection{
  \frame{\subsectionpage}
}
\setlength{\emergencystretch}{3em}  % prevent overfull lines
\providecommand{\tightlist}{%
  \setlength{\itemsep}{0pt}\setlength{\parskip}{0pt}}
\setcounter{secnumdepth}{0}

% set default figure placement to htbp
\makeatletter
\def\fps@figure{htbp}
\makeatother

% header.tex: boring LaTeX/Beamer details + macros

% math shorthand
\usepackage{bm}
\usepackage{amstext}
\usepackage{amsthm}
\usepackage{amsmath}
\usepackage{mathtools}
\newcommand{\R}{\mathbb{R}}
\newcommand{\D}{\mathcal{D}}
\newcommand{\E}{\mathbb{E}}
\newcommand{\I}{\mathbb{I}}
\newcommand{\pr}{\mathbb{P}}
\newcommand{\F}{\mathcal{F}}
\newcommand{\X}{\mathcal{X}}
\newcommand{\M}{\mathcal{M}}
\newcommand{\lik}{\mathcal{L}}

\title{Causal Inference in Biomedical Research}
\providecommand{\subtitle}[1]{}
\subtitle{Methods Discussion, California Department of Public Health}
\author{\href{https://nimahejazi.org}{Nima Hejazi}, UC Berkeley}
\date{2019-09-18}

\begin{document}
\frame{\titlepage}

\begin{frame}
\tableofcontents[hideallsubsections]
\end{frame}
\hypertarget{causal-inference-as-a-framework}{%
\section{Causal Inference as a
Framework}\label{causal-inference-as-a-framework}}

\begin{frame}{Assessing HIV vaccine efficacy}
\protect\hypertarget{assessing-hiv-vaccine-efficacy}{}

\begin{itemize}[<+->]
\tightlist
\item
  The HIV Vaccine Trials Network (HVTN) is a large-scale collaboration
  leading efforts to design and verify the efficacy of candidate
  preventive vaccines.
\item
  As an example, consider a study in which a vaccine (\(A\)) is
  administered, based on baseline covariates \(L\) (e.g., sex, age),
  and, eventually, infection status (\(Y\)) is measured.
\item
  We'd like to assess the efficacy of the vaccine in preventing
  \(Y = 1\) (HIV infection). How do we go about doing this?
\item
  What about the baseline covariates \(L\)?
\end{itemize}

\end{frame}

\begin{frame}{Causal effects and interventions}
\protect\hypertarget{causal-effects-and-interventions}{}

\begin{itemize}[<+->]
\tightlist
\item
  Formally, we observe data \(O = (Y, A, L)\): an outcome \(Y\), a
  treatment (or exposure) \(A\), and baseline covariates \(L\).

  \begin{itemize}[<+->]
  \tightlist
  \item
    To simplify, consider \(A \in \{0, 1\}\) -- i.e., two treatment
    levels.
  \item
    \emph{Goal:} What happens to the outcome \(Y\) when we set \(A\) to
    \(1\)? Note that this implies an \emph{intervention} on the system.
  \end{itemize}
\item
  Now, \(A\) and \(Y\) have a common cause (namely, \(L\)), so how to do
  we account for this information?

  \begin{itemize}[<+->]
  \tightlist
  \item
    When we can randomize, we randomize.
  \item
    So, are observational studies hopeless?
  \end{itemize}
\end{itemize}

\end{frame}

\begin{frame}{Causal effects and potential outcomes}
\protect\hypertarget{causal-effects-and-potential-outcomes}{}

\begin{itemize}[<+->]
\tightlist
\item
  Although we see \(O = (Y, A, L)\), we really would like to have seen
  an unmasked version of this data \(O = (Y(1), Y(0), A, L)\).

  \begin{itemize}[<+->]
  \tightlist
  \item
    \(\{(Y(1), Y(0)\}\) are referred to as \emph{potential outcomes}.

    \begin{itemize}[<+->]
    \tightlist
    \item
      \(Y(1)\): What would have \(Y\) been had \(A = 1\).
    \item
      Similarly, \(Y(0)\) corresponds to the case of \(A = 0\).
    \end{itemize}
  \item
    Intervening on the observed system -- e.g., setting \(A\) to \(1\)
    -- is a thought experiment asking ``what is \(Y(1)\)?''
  \end{itemize}
\item
  Interventions are hypothetical disruptions to the observed system,
  defining various potential outcomes.
\item
  In our example, \(Y(1)\) is HIV infection status had the unit received
  the vaccine.
\end{itemize}

\end{frame}

\begin{frame}{Potential outcomes and treatment effects}
\protect\hypertarget{potential-outcomes-and-treatment-effects}{}

\begin{itemize}[<+->]
\tightlist
\item
  If we had \(\{(Y(1), Y(0)\}\), we could define the \emph{treatment
  effect} \(\tau = Y(1) - Y(0)\) for an observed unit.
\item
  We consider \(\tau = Y(1) - Y(0)\) across all units in the study to
  define the sample treatment effect.
\item
  The intervention \(A = 1\) breaks the relationship between \(A\) and
  \(L\) so, in defining \(\tau\), we need not worry about \(L\).
\item
  What if we had assigned vaccination based on clinical information
  (e.g., behavioral risk of HIV infection)?
\end{itemize}

\end{frame}

\begin{frame}{Identification of causal effects}
\protect\hypertarget{identification-of-causal-effects}{}

\begin{itemize}[<+->]
\tightlist
\item
  In order to establish an estimator for a causal effect, we need
  conditions under which the causal quantity matches a statistically
  estimable quantity.
\item
  The set of conditions for this process are \emph{identification}
  results, which vary with types of interventions and effects of
  interest.
\item
  In the most classical cases, we need four assumptions:

  \begin{enumerate}[<+->]
  \tightlist
  \item
    Consistency: ``no other versions of treatment.''
  \item
    No unobserved confounders: ``we have enough information to
    disentangle potential outcomes.''
  \item
    No interference: ``your treatment does not affect my outcome.''
  \item
    Positivity: ``it must be possible to receive treatment.''
  \end{enumerate}
\end{itemize}

\end{frame}

\begin{frame}{Assessing causal effects}
\protect\hypertarget{assessing-causal-effects}{}

\begin{itemize}[<+->]
\tightlist
\item
  In practice, clinicians make treatment decisions, thus undoing are
  decoupling of \(L\) and \(A\). What now?
\item
  The \emph{propensity score} \(f(A \mid L)\) is the probability of
  receiving treatment, given observed covariates.
\item
  This quantity can be used to construct valid estimators of causal
  effects, since it contains all information about the relationship
  between \(A\) and \(L\).
\item
  In a randomized study, the propensity score contains no information,
  i.e., \(f(A \mid L) = 0.5\).
\end{itemize}

\end{frame}

\begin{frame}{What about censoring?}
\protect\hypertarget{what-about-censoring}{}

\begin{itemize}[<+->]
\tightlist
\item
  In considering potential outcomes, we've been dealing with censoring,
  so it's naturally built into our framework.
\item
  Subject to censoring, the observed data is \(O = (\Delta, Y, A, L)\),
  where \(\Delta \in \{0,1\}\) indicates missingness.
\item
  To assess causal effects in the presence of censoring, consider a
  \emph{joint intervention}, e.g., \(\{A = 1, \Delta = 1\}\) that makes
  a unit both treated \emph{and} observed.
\item
  Constructing estimators involves assessing the mechanism of
  missingness but the problem is naturally handled.
\end{itemize}

\end{frame}

\begin{frame}[fragile]{Targeted Learning: Causal inference to
statistics}
\protect\hypertarget{targeted-learning-causal-inference-to-statistics}{}

\begin{itemize}[<+->]
\tightlist
\item
  Targeted Learning is a subfield of statistics developing robust
  estimators of scientifically meaningful parameters.
\item
  In general, such parameters are inspired by causal quantities, with
  robust estimators constructed using machine learning.
\item
  Our group has pioneered a user-friendly software ecosystem,
  \href{https://tlverse.org}{the \texttt{tlverse}}, to accommodate data
  analytic applications.
\item
  Check out our software and publicly available materials:

  \begin{itemize}[<+->]
  \tightlist
  \item
    The \texttt{tlverse} homepage: \url{https://tlverse.org}
  \item
    The \texttt{tlverse} handbook:
    \url{https://tlverse.org/tlverse-handbook}
  \item
    Recent workshop: \url{https://tlverse.org/acic2019-workshop}
  \end{itemize}
\item
  The \texttt{tlverse} is a team effort, with four founding developers,
  and several PhD students now supporting continued work.
\end{itemize}

\end{frame}

\hypertarget{estimation-the-g-formula}{%
\section{Estimation: The G-Formula}\label{estimation-the-g-formula}}

\begin{frame}{The g-formula at a single time point}
\protect\hypertarget{the-g-formula-at-a-single-time-point}{}

\begin{itemize}[<+->]
\tightlist
\item
  Under identifiability conditions, the counterfactual mean
  \(\E[Y^{A=a}]\) is a weighted average of the mean outcome,
  standardized for observed covariates in the study population:
  \[\sum_{l_1} \E[Y \mid A_1 = a_1, L_1 = l_1]f(l_1),\] where
  \(f(l_1) = \pr(L_1 = l_1)\).
\item
  In practice, \(\E[Y \mid A_1 = a_1, L_1 = l_1]\) must be learned from
  the data, e.g., with a parametric regression.
\item
  Consistency of the g-formula estimator depends on correct
  specification of this model for the outcome mechanism.
\end{itemize}

\end{frame}

\begin{frame}{The g-formula across multiple time points}
\protect\hypertarget{the-g-formula-across-multiple-time-points}{}

\begin{itemize}[<+->]
\tightlist
\item
  The g-formula may straightforwardly be extended to the case of
  multiple time points, again as a simple weighted average:
  \[\sum_{l_1} \E[Y \mid A_0 = a_0, A_1 = a_1, L_1 = l_1]f(l_1 \mid a_0),\]
  for a simple treatment strategy across two time points.
\item
  Key difference from the point treatment case: every factor is
  conditional on its past treatment and covariate history.
\item
  Positivity is required, i.e., \(f(l_1 \mid a_0) \neq 0\).
\end{itemize}

\end{frame}

\begin{frame}{The g-formula across multiple time points}
\protect\hypertarget{the-g-formula-across-multiple-time-points-1}{}

\begin{itemize}[<+->]
\tightlist
\item
  In the case of multiple time points \(k\), the g-formula is
  \[\sum_{\bar{l}} \E[Y \mid \bar{A} = \bar{a}, \bar{L} = \bar{l}]
    \prod_{k=0}^{K} f(l_k \mid \bar{a}_{k-1}, \bar{l}_{k-1}).\]
\item
  In practice, we estimate the components
  \(\hat{\E}[]Y \mid \bar{A} = \bar{a}, \bar{L} = \bar{l}]\) and
  \(\hat{f}(l_k \mid \bar{a}_{k-1}, \bar{l}_{k-1})\), i.e., the
  \emph{plug-in g-formula}.
\item
  When the components are fit with parametric models, this plug-in
  g-formula estimator is referred to as the \emph{parametric g-formula}.
\end{itemize}

\end{frame}

\hypertarget{estimation-inverse-probability-weighting}{%
\section{Estimation: Inverse Probability
Weighting}\label{estimation-inverse-probability-weighting}}

\begin{frame}{Inverse probability weighting at a single time point}
\protect\hypertarget{inverse-probability-weighting-at-a-single-time-point}{}

\begin{itemize}[<+->]
\tightlist
\item
  To assess the counterfactual mean outcome \(\E[Y^{A_1 = a}]\) under a
  treatment contrast \(a\), it is sufficient to examine the
  counterfactual mean in the pseudo-population based on the weighting
  scheme \(W^{A_1} = \frac{1}{f(A_1 \mid L_1)}\).
\item
  The inverse weight \(W^A\) is constructed based on the propensity
  score \(f(A_1 \mid L_1)\).
\item
  One may also use stabilized weights
  \(SW^{A_1} = \frac{f(A_1)}{f(A_1 \mid L_1)}\).
\end{itemize}

\end{frame}

\begin{frame}{Inverse probability weighting across multiple time points}
\protect\hypertarget{inverse-probability-weighting-across-multiple-time-points}{}

\begin{itemize}[<+->]
\tightlist
\item
  Consider now just two time points where we have time-varying
  treatments \(\bar{A} = (A_0, A_1)\) and covariates
  \(\bar{L} = (L_0, L_1)\).
\item
  IP weights must be extended to account for history:
  \[W^{\bar{A}} = \prod_{k=0}^K \frac{1}{f(A_k \mid \bar{A}_{k-1},
  \bar{L}_k)}.\]
\item
  The denominator may be interpreted as the individual's probability of
  receiving their observed treatment history, conditional on their
  covariate history.
\item
  With nonstabilized IP weights, a pseudo-population is created in which
  randomization probabilities at each time \(k\) are constant, so
  sequential exchangeability holds.
\end{itemize}

\end{frame}

\begin{frame}{Inverse probability weighting across multiple time points}
\protect\hypertarget{inverse-probability-weighting-across-multiple-time-points-1}{}

\begin{itemize}[<+->]
\tightlist
\item
  In general, the \emph{stabilized} IP weights are
  \[SW^{\bar{A}} = \prod_{k=0}^K \frac{f(A_k \mid \bar{A}_{k-1})}
  {f(A_k \mid \bar{A}_{k-1}, \bar{L}_k)}.\]
\item
  Note that the stabilized term (in the numerator) may be misspecified
  without inducing any bias in the estimate of the counterfactual mean.
\item
  With stabilized IP weights, a pseudo-population is created in which
  randomization probabilities at each time \(k\) are dependent only on
  past treatment history, so sequential exchangeability holds.
\end{itemize}

\end{frame}

\begin{frame}{Inverse probability weighting across multiple time points}
\protect\hypertarget{inverse-probability-weighting-across-multiple-time-points-2}{}

\begin{itemize}[<+->]
\tightlist
\item
  In observational studies, \(f(A_k \mid \bar{A}_{k-1}, \bar{L}_k)\)
  must be estimated (known by design in randomized trials).
\item
  When comparing parametric g-formula and IP weighting estimates,
  substantial difference implies misspecification of either the outcome
  or treatment assignment model.
\item
  Importantly, while a difference implies model misspecification,
  agreement between the two \emph{does not certify} their correctness.
\item
  When fitting an MSM (e.g.,
  \(\E[Y \mid \bar{A}] = \theta_0 + \theta_1 \sum_{k=0}^K A_k\)), use
  weighted least squared with IP weights.
\item
  In the case of the IP-weighted MSM, use of the stabilized weights
  \(SW^{\bar{A}}\) generally gives narrower confidence intervals.
\end{itemize}

\end{frame}

\hypertarget{censoring-coping-with-missingness}{%
\section{Censoring: Coping with
Missingness}\label{censoring-coping-with-missingness}}

\begin{frame}{Censoring at a single time point}
\protect\hypertarget{censoring-at-a-single-time-point}{}

\begin{itemize}[<+->]
\tightlist
\item
  Let \(C = 0\) denote being uncensored. Recall that conditioning on
  being uncensored induces selection bias under the null.
\item
  When \(C\) is either a collider on a pathway between treatment \(A\)
  and outcome \(Y\) or the descendent of such a collider.
\item
  We address censoring by considering a \emph{joint intervention}
  \(do(A = a, C = 0)\) that makes an individual uncensored and sets
  their treatment value to \(a\).
\end{itemize}

\end{frame}

\begin{frame}{Addressing censoring across multiple time points}
\protect\hypertarget{addressing-censoring-across-multiple-time-points}{}

\begin{itemize}[<+->]
\tightlist
\item
  Censoring is monotonic, i.e., \(C_m = 0\) implies
  \(C_j = 0 \quad \forall \quad j \in \{1, \ldots, m-1\}\).
\item
  As before, we consider a joint intervention:
  \((\bar{a}, \bar{c} = 0)\) that sets the treatment history to
  \(\bar{a}\) and induces no censoring.
\item
  G-formula for the counterfactual mean \(\E[Y^{\bar{a}}]\) is
  \[\sum_{\bar{l}} \E[Y \mid \bar{A} = \bar{a}, \bar{C} = \bar{0}, \bar{L} =
  \bar{l}] \prod_{k=0}^K f(l_k \mid \bar{a}_{k-1}, c_{k-1} = 0,
  \bar{l}_{k-1}).\]
\end{itemize}

\end{frame}

\begin{frame}{Addressing censoring across multiple time points}
\protect\hypertarget{addressing-censoring-across-multiple-time-points-1}{}

\begin{itemize}[<+->]
\tightlist
\item
  IP-weighting may also be used to estimate the counterfactual mean
  using a pseudo-population created by the weights
  \(W^{\bar{A}} \times W^{\bar{C}}\), where
  \[W^{\bar{C}} = \prod_{k=1}^{K+1} \frac{1}{\pr(C_k = 0 \mid
  \bar{A}_{k-1}, C_{k-1} = 0, \bar{L}_k)}.\]
\item
  The effect of IP-weighting to create a pseudo-population is as before
  --- censoring is abolished by replacing censored individuals with
  uncensored copies with the same covariate history and treatment
  history. Thus, the pseudo-population is of the same size as the
  original \emph{before} censoring.
\end{itemize}

\end{frame}

\begin{frame}{Addressing censoring across multiple time points}
\protect\hypertarget{addressing-censoring-across-multiple-time-points-2}{}

\begin{itemize}[<+->]
\tightlist
\item
  Stabilized IP weights also create a (different) pseudo-population,
  where the weight \(W^{\bar{C}}\) is \[W^{\bar{C}} = \prod_{k=1}^{K+1}
  \frac{\pr(C_k = 0 \mid \bar{A}_{k-1}, C_{k-1} = 0)}
  {\pr(C_k = 0 \mid \bar{A}_{k-1}, C_{k-1} = 0, \bar{L}_k)}.\]
\item
  Stabilized weights work a bit differently: this pseudo-population is
  of the same size as the original \emph{after} censoring but with the
  censoring event made random across observed times, thus removing any
  selection bias.
\end{itemize}

\end{frame}

\end{document}
